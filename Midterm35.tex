\documentclass[10pt, addpoints] {exam}
\usepackage[margin=0.75in]{geometry}
\usepackage{color} %green = just honors, red = just standard
\usepackage{amssymb,amsmath}
\usepackage{graphicx}


\begin{document}

\title{Honors Physics Midterm Exam}
\author{Ms. Barry \\ A316 \\ abarryqhs@gmail.com}
\date{January 2014}

\pagestyle{headandfoot}
\runningfootrule
\firstpageheader{}{}{}
\runningheader{}{}{}
\firstpagefooter{}{}{}
\runningfooter{Honors Physics}{Midterm Exam}{Page \thepage\ of \numpages}

\maketitle

Respond fully to each of the following multiple choice and open response questions to the best of your ability. Read all questions carefully, and be sure to answer all parts of multiple-part questions. Write all your answers on your answer sheet. \textbf{Do not write on this exam.} You may find the following equations helpful:
$$ s = d/t $$
%$$ v = d/t  $$
$$ a = \Delta v/ t $$
$$ d = v_it + 1/2at^2$$
$$ v^2_f = v^2_i + 2ad$$
$$v_f = v_i + at$$
$$ F = ma $$
$$ f = \mu N$$
$$ p = mv $$
$$ J = \Delta p = Ft$$
$$ W = Fd $$
$$ P = W/t $$
$$ PE_g = mgh $$
$$ PE_{spring} = 1/2 kx^2$$
$$ KE = 1/2 mv^2 $$

\includegraphics[scale=0.65]{cartoon1.png}
\centering



\newpage
\raggedright
\textbf{Part A:} Matching\\
\textit{For each of the following quantities, select the appropriate unit and whether it is defined as a vector or a scalar quantity:}
\\[0.2in]

\centering
\begin{tabular}{ l || l | l | l }
  %\hline
   quantity & unit & Enter 'A' for scalar and 'B' for vector  & \\                   
  \hline
  mass 			& 1.  		& 15.		& A. m\\
  time 			& 2. 		& 16.		& B. m/s\\
  distance 			& 3. 		& 17.		& C. m/s$^2$\\
  displacement 		& 4. 		& 18.		& D. kgm/s\\
  speed 			& 5. 		& 19.		& E. N\\ 	
  velocity			& 6.  		& 20.		& F. W\\
  acceleration		& 7.  		& 21.		& G. J\\
  force			& 8.  		& 22.		& H. kg\\
  momentum		& 9.  		& 23.		& I. s\\
  impulse			& 10.  	& 24.		& \\
  work 			& 11.  	& 25.		& \\
  power			& 12.  	& 26.		& \\
  potential energy	& 13.  	& 27.		& \\
  kinetic energy		& 14.  	& 28.		& \\
  
  \hline  
\end{tabular}
\\[0.5in]

\raggedright
\textbf{Part B:} Multiple Choice \\
\textit{Select the best answer to each question. Be sure to answer all questions!} \\

\begin{questions}
%\setcounter{question}{28}

\question
Which of the following is closest in length to a yard?

\begin {choices}
\choice 0.01m
\choice 0.1m
\correctchoice 1m
\choice 10m
\choice 100m
\end {choices}

\question
A gram is:

\begin {choices}
\choice $10^{-6}$ kg
\correctchoice $10^{-3}$ kg
\choice 1 kg
\choice $10^3$ kg
\choice $10^6$ kg
\end {choices}

\question
(5.0 x $10^4$) x (3.0 x $10^{-6}$) =

\begin {choices}
\choice 1.5 x $10^{-3}$
\correctchoice 1.5 x $10^{-1}$
\choice 1.5 x $10^1$
\choice 1.5 x $10^3$
\choice 1.5 x $10^5$
\end{choices}


%%\textcolor{green}
%\question
%(7.0 x $10^6$)/(2.0 x $10^{-6}) = %Solve \frac{a}{b} %Solve \frac{7.0 x $10^6$}{2.0 x $10^{-6}$}
%\begin {oneparchoices}
%\choice 3.5 x $10^{-12}$
%\choice 3.5 x $10^{-6}$
%\choice 3.5
%\choice 3.5 x $10^6$
%\correctchoice 3.5 x $10^{12}$
%\end{oneparchoices}


\question
1 mi is equivalent to 1609 m so 55 mph is:

\begin {choices}
\choice 15 m/s
\correctchoice 25 m/s
\choice 66 m/s
\choice 88 m/s
\choice 1500 m/s
\end {choices}


\question
A particle moves along the x axis from $x_i$ to $x_f$ . Of the following values of the initial and final coordinates, which results in the displacement with the largest magnitude?

\begin {choices}
\choice $x_i$ = 4m, $x_f$ = 6m
\choice $x_i$ = -4m, $x_f$ = -8m
\choice $x_i$ = -4m, $x_f$ = 2m
\choice $x_i$ = 4m, $x_f$ = -2m
\correctchoice $x_i$ = -4m, $x_f$ = 4m
\end {choices}

\question
The average speed of a moving object during a given interval of time is always:
\begin {choices}
\choice the magnitude of its average velocity over the interval
\correctchoice the distance covered during the time interval divided by the time interval
\choice one-half its speed at the end of the interval
\choice its acceleration multiplied by the time interval
\choice one-half its acceleration multiplied by the time interval.
\end {choices} 

%\textcolor{red}{
%\question
%A car starts from Hither, goes 50 km in a straight line to Yon, immediately turns around, and returns to Hither. The time for this round trip is 2 hours. The magnitude of the \textbf{average speed} of the car for this round trip is:
%\begin {choices}
%\correctchoice 0
%\choice 50 km/hr
%\choice 100 km/hr
%\choice 200 km/hr
%\choice cannot be calculated without the acceleration
%\end {choices}}


\question
A car starts from Hither, goes 50 km in a straight line to Yon, immediately turns around, and returns to Hither. The time for this round trip is 2 hours. The magnitude of the \textbf{average velocity} of the car for this round trip is:

\begin {choices}
\correctchoice 0
\choice 50 km/hr
\choice 100 km/hr
\choice 200 km/hr
\choice cannot be calculated without the acceleration
\end {choices}



\question
Of the following situations, which one is impossible?
\begin {choices}
\choice A body having velocity east and acceleration east
\choice A body having velocity east and acceleration west
\choice A body having zero velocity and non-zero acceleration
\choice A body having constant acceleration and variable velocity
\correctchoice A body having constant velocity and variable acceleration
\end {choices}


%\textcolor{green}{
\question
A racing car traveling with constant acceleration increases its speed from 10m/s to 50m/s over a distance of 60m. How long does this take?
\\
\begin {choices}
\correctchoice 2.0 s
\choice 4.0 s
\choice 5.0 s
\choice 8.0 s
\choice The time cannot be calculated since the speed is not constant
\end {choices}


\question
A car starts from rest and goes down a slope with a constant acceleration of $5 m/s^2$. After 5s the car reaches the bottom of the hill. Its speed at the bottom of the hill is:

\begin {choices}
\choice 1 m/s
\choice 12.5 m/s
\correctchoice 25 m/s
\choice 50 m/s
\choice 160 m/s
\end {choices}

\newpage
\question
A ball is in free fall. Its acceleration is:

\begin {choices}
\correctchoice downward during both ascent and descent
\choice downward during ascent and upward during descent
\choice upward during ascent and downward during descent
\choice upward during both ascent and descent
\choice downward at all times except at the very top, when it is zero
\end{choices}


%\textcolor{green}{
\question
Which one of the following statements is correct for a free-falling object released from rest?
\\
\begin {choices}
\correctchoice The average velocity during the first second of time is 4.9m/s
\choice During each second the object falls 9.8m
\choice The acceleration changes by $9.8m/s^2$ every second
\choice The object falls 9.8m during the first second of time
\choice The acceleration of the object is proportional to its weight
\end{choices}

%\textcolor{red}{
%\question
%A freely falling body has a constant acceleration of 9.8 m/s$^2$. This means that:
%\begin {choices}
%\choice the body falls 9.8 m during each second
%\choice the body falls 9.8 m during the first second only
%\correctchoice the speed of the body increases by 9.8 m/s during each second
%\choice the acceleration of the body increases by 9.8 m/s2 during each second
%\choice the acceleration of the body decreases by 9.8 m/s2 during each second
%\end {choices}}


\question
An object is thrown vertically upward at 35 m/s. The velocity of the object 5 s later is approximately:
\begin {choices}
\choice 7.0 m/s up
\choice 15 m/s down
\choice 15 m/s up
\correctchoice 85 m/s down
\choice 85 m/s up
\end{choices}


\question
Which of the following five coordinate versus time graphs represents the motion of an object moving with a constant nonzero speed?

\includegraphics[scale = 0.5] {Picture4}
\centering



\raggedright
%\textcolor{green}{
\question
Which of the following five acceleration versus time graphs is correct for an object moving in
a straight line at a constant velocity of 20 m/s?

\includegraphics{graphs2.jpg}
\centering

\raggedright
\question
The diagram shows a \textbf{velocity-time} graph for a car moving in a straight line. At point Q the car must be:

\includegraphics {pandq.png}
\centering

\raggedright
\begin{choices}
\choice moving with zero acceleration
\choice traveling downhill
\choice traveling below ground-level
\choice reducing speed
\correctchoice traveling in the reverse direction to that at point P
\end {choices}

\raggedright

\question At point P the car must be:
\begin{choices}
\choice moving with zero acceleration
\choice climbing the hill
\correctchoice accelerating
\choice stationary
\choice moving at about 45 degrees with respect to the x axis
\end{choices}

\textit{The following graph represents the straight-line motion of a car.}

\includegraphics{vtgraph.png}
\centering

\raggedright
%\textcolor{green}{
\question How far does the car travel between t=2s and t=5s?
\\
\begin {choices}
\choice 4m
\choice 12m
\choice 24m
\correctchoice 36m
\choice 60m
\end{choices}

\question
Which of the following statements is true?
\\
\begin{choices}
\choice The car accelerates, stops, and reverses
\correctchoice The car accelerates at 6 m/s$^2$ for the first 2s
\choice The car is moving for a total time of 12 s
\choice The car decelerates at 12 m/s$^2$ for the last 4s
\choice The car returns to its starting point when t = 9s
\end{choices}

%ANOTHER p-t graph for Standard?
\newpage
\question
A vector of magnitude 20 is added to a vector of magnitude 25. The magnitude of this sum might be:
\\
\begin{choices}
\choice 0
\choice 3
\correctchoice 12
\choice 47
\choice 50
\end {choices}


\question
The vector $-\vec{A}$ is:
\begin{choices}
\choice greater than $\vec{A}$ in magnitude
\choice less than $\vec{A}$ in magnitude
\choice in the same direction as $\vec{A}$
\correctchoice in the direction opposite to $\vec{A}$
\choice perpendicular to $\vec{A}$
\end{choices}


%\question
%Velocity is defined as:
%\begin{choices}
%\correctchoice rate of change of position with time
%\choice position divided by time
%\choice rate of change of acceleration with time
%\choice a speeding up or slowing down
%\choice change of position
%\end{choices}
%
%\question
%Acceleration is defined as:
%\begin{choices}
%\choice rate of change of position with time
%\choice speed divided by time
%\correctchoice rate of change of velocity with time
%\choice a speeding up or slowing down
%\choice change of velocity
%\end{choices}


\question
A stone thrown at an angle from the top of a tall building follows a path that is:
\\
\begin{choices}
\choice circular
\choice linear
\choice hyperbolic
\correctchoice parabolic
\choice proportional
\end{choices}

\question 
A stone is thrown horizontally and follows the path XYZ shown. The direction of acceleration at point Y is:
\includegraphics[scale=0.5]{gravity2.png}

\question
A bullet shot horizontally from a gun:
\begin{choices}
\choice strikes the ground much later than one dropped vertically from the same point at the same instant
\choice never strikes the ground
\correctchoice strikes the ground at approximately the same time as one dropped vertically from the same point at the same instant
\choice travels in a straight line
\choice strikes the ground much sooner than one dropped from the same point at the same instant
\end{choices}

\newpage
\question
A bomber flying in level flight with constant velocity releases a bomb before it is over the target. Neglecting air resistance, which one of the following is NOT true?
\begin{choices}
\choice The bomber is over the target when the bomb strikes
\choice The acceleration of the bomb is constant
\correctchoice The horizontal velocity of the plane equals the vertical velocity of the bomb when it hits the target
\choice The bomb travels in a curved path
\choice The time of flight of the bomb is independent of the horizontal speed of the plane
\end{choices}

%\textcolor{green}{
%%ADD PICTURE
%\question
%An object is shot from the back of a railroad flatcar moving at 40km/h on a straight horizontal road. The launcher is aimed upward, perpendicular to the bed of the flatcar. The object falls:
%\begin {choices}
%\choice in front of the flatcar
%\choice behind the flatcar
%\correctchoice on the flatcar
%\choice either behind or in front of the flatcar, depending on the initial speed of the object
%\choice to the side of the flatcar
%\end {choices}}

\question
In metric system units a force is numerically equal to the \rule{1in}{.1pt}, when the force is applied to it.
\begin{choices}
\choice velocity of the standard kilogram
\choice speed of the standard kilogram
\choice velocity of any object
\correctchoice acceleration of the standard kilogram
\choice acceleration of any object
\end{choices}

%A river boat question??

\question
A force of 1N is:
\\
\begin{choices}
\choice 1 kg/s
\choice 1 kg$\cdot$m/s
\correctchoice 1 kg$\cdot$m/s$^2$
\choice 1 kg$\cdot$m$^2$/s
\choice 1 kg$\cdot$m$^2$/s$^2$
\end{choices}

\question
Acceleration is always in the direction:
\\
\begin{choices}
\choice of the displacement
\choice of the initial velocity
\choice of the final velocity
\correctchoice of the net force
\choice opposite to the frictional force
\end{choices}

%\textcolor{green}{
\question
The term �mass� refers to the same physical concept as:
\\
\begin{choices}
\choice weight
\correctchoice inertia
\choice force
\choice acceleration
\choice volume
\end{choices}

\question
Mass differs from weight in that:
\begin{choices}
\choice all objects have weight but some lack mass
\correctchoice weight is a force and mass is not
\choice the mass of an object is always more than its weight
\choice mass can be expressed only in the metric system
\choice there is no difference
\end{choices}

%spring scale operation question

\newpage
\question
A feather and a lead ball are dropped from rest in vacuum on the Moon. The acceleration of the feather is:
\begin{choices}
\choice more than that of the lead ball
\correctchoice the same as that of the lead ball
\choice less than that of the lead ball
\choice 9.8m/s$^2$
\choice zero since it floats in a vacuum
\end{choices}

\question
The block shown moves with constant velocity on a horizontal surface. Two of the forces on it are shown. A frictional force exerted by the surface is the only other horizontal force on the block. The frictional force is:
\includegraphics{3N5Nbox.png}
\centering
\begin{choices}
\choice 0
\correctchoice 2N, leftward
\choice 2N, rightward
\choice slightly more than 2N, leftward
\choice slightly less than 2N, leftward
\end{choices}

\raggedright
\question
A crate rests on a horizontal surface and a woman pulls on it with a 10-N force. Rank the situations shown below according to the magnitude of the normal force exerted by the surface on the crate, least to greatest.
\includegraphics{3boxes.png}
\centering
\begin{choices}
\choice 1, 2, 3
\choice 2, 1, 3
\choice 2, 3, 1
\choice 1, 3, 2
\correctchoice 3, 2, 1
\end{choices}

\raggedright
\question
Equal forces act on isolated bodies A and B. The mass of B is three times that of A. The magnitude of the acceleration of A is:
\begin{choices}
\correctchoice three times that of B
\choice 1/3 that of B
\choice the same as B
\choice nine times that of B
\choice 1/9 that of B
\end{choices}

\question
A car travels east at constant velocity. The net force on the car is:
\\
\begin{choices}
\correctchoice east
\choice west
\choice up
\choice down
\choice zero
\end{choices}

\newpage
\question
A 400-N steel ball is suspended by a light rope from the ceiling. The tension in the rope is:
\\
\begin{choices}
\correctchoice 400N
\choice 800N
\choice zero
\choice 200N
\choice 560N
\end{choices}

%\textcolor{green}{
\question
A man weighing 700 N is in an elevator that is accelerating upward at 4m/s$^2$. The force exerted on him by the elevator floor is:
\\
\begin{choices}
\choice 71N
\choice 290N
\choice 410N
\choice 700N
\correctchoice 990N
\end{choices}

\question
The reaction force does not cancel the action force because:
\begin{choices}
\choice the action force is greater than the reaction force
\correctchoice they act on different bodies
\choice they act in the same direction
\choice the reaction force exists only after the action force is removed
\choice the reaction force is greater than the action force
\end{choices}

\question
A lead block is suspended from your hand by a string. The reaction to the force of gravity on the block is the force exerted by:
\begin{choices}
\choice the string on the block
\choice the block on the string
\choice the string on the hand
\choice the hand on the string
\correctchoice the block on Earth
\end{choices}

\question
A brick slides on a horizontal surface. Which of the following will increase the magnitude of the frictional force on it?
\begin{choices}
\correctchoice Putting a second brick on top
\choice Decreasing the surface area of contact
\choice Increasing the surface area of contact
\choice Decreasing the mass of the brick
\choice None of the above
\end{choices}

\question
Why do raindrops fall with constant speed during the later stages of their descent?
\begin{choices}
\choice The gravitational force is the same for all drops
\correctchoice Air resistance just balances the force of gravity
\choice The drops all fall from the same height
\choice The force of gravity is negligible for objects as small as raindrops
\choice Gravity cannot increase the speed of a falling object to more than 9.8m/s
\end{choices}

\newpage
\question
The momentum of an object at a given instant is independent of its:
\begin{choices}
\choice inertia
\choice mass
\choice speed
\choice velocity
\correctchoice acceleration
\end{choices}

\question
Two objects, P and Q, have the same momentum. Q has more kinetic energy than P if it:
\begin{choices}
\correctchoice weighs more than P
\choice is moving faster than P
\choice weighs the same as P
\choice is moving slower than P
\choice is moving at the same speed as P
\end{choices}

\question
A 1.0-kg ball moving at 2.0m/s perpendicular to a wall rebounds from the wall at 1.5m/s. The change in the momentum of the ball is:
\begin{choices}
\choice zero
\choice 0.5Ns away from wall
\choice 0.5Ns toward wall
\correctchoice 3.5Ns away from wall
\choice 3.5Ns toward wall
\end{choices}

\question
When you step on the accelerator to increase the speed of your car, the force that accelerates the car is:
\begin{choices}
\choice the force of your foot on the accelerator
\correctchoice the force of friction of the road on the tires
\choice the force of the engine on the drive shaft
\choice the normal force of the road on the tires
\choice none of the above
\end{choices}

\question
A 2.5-kg stone is released from rest and falls toward Earth. After 4.0 s, the magnitude of its momentum is:
\begin{choices}
\correctchoice 98 kgm/s
\choice 78 kgm/s
\choice 39 kgm/s
\choice 24 kgm/s
\choice zero
\end{choices}

\question
A man is marooned at rest on level frictionless ice. In desperation, he hurls his shoe to the right at 15m/s. If the man weighs 720N and the shoe weighs 4.0N, the man moves to the left with a speed of:
\begin{choices}
\choice 0
\choice 2.1 x $10^{-2}$m/s
\correctchoice 8.3 x $10^{-2}$m/s
\choice 15m/s
\choice 2.7 x $10^3$m/s
\end{choices}

\newpage
\question
A projectile in flight explodes into several fragments. The total momentum of the fragments immediately after this explosion:
\begin{choices}
\correctchoice is the same as the momentum of the projectile immediately before the explosion
\choice has been changed into kinetic energy of the fragments
\choice is less than the momentum of the projectile immediately before the explosion
\choice is more than the momentum of the projectile immediately before the explosion
\choice has been changed into radiant energy
\end{choices}

\question
The thrust of a rocket is:
\begin{choices}
\choice a gravitational force acting on the rocket
\correctchoice the force of the exiting fuel gases on the rocket
\choice any force that is external to the rocket-fuel system
\choice a force that arises from the reduction in mass of the rocket-fuel system
\choice none of the above
\end{choices}

\question
A 1000-kg space probe is motionless in space. To start moving, its main engine is fired for 5 s during which time it ejects exhaust gases at 5000m/s. At the end of this process it is moving at 20m/s. The approximate mass of the ejected gas is: \\
\includegraphics[scale=0.5]{rocket.png}
\centering
\begin{choices}
\choice 0.8kg
\correctchoice 4 kg
\choice 5 kg
\choice 20 kg
\choice 25 kg
\end{choices}

\raggedright
\question
The physical quantity �impulse� has the same dimensions as that of:
\begin{choices}
\choice force
\choice power
\choice energy
\correctchoice momentum
\choice work
\end{choices}

\question
A student�s life was saved in an automobile accident because an airbag expanded in front of his head. If the car had not been equipped with an airbag, the windshield would have stopped the motion of his head in a much shorter time. Compared to the windshield, the airbag:
\begin{choices}
\choice causes a much smaller change in momentum
\choice exerts a much smaller impulse
\choice causes a much smaller change in kinetic energy
\correctchoice exerts a much smaller force
\choice does much more work
\end{choices}

\newpage
\question
Whenever an object strikes a stationary object of equal mass:
\begin{choices}
\choice the two objects cannot stick together
\choice the collision must be elastic
\choice the first object must stop
\choice momentum is not necessarily conserved
\correctchoice none of the above
\end{choices}

%baseball compression

\question
Which of the following is NOT a correct unit for work?
\begin{choices}
\choice erg
\choice ft$\cdot$lb
\correctchoice watt
\choice newton$\cdot$meter
\choice joule
\end{choices}

\question
A boy holds a 40-N weight at arm�s length for 10 s. His arm is 1.5m above the ground. The
work done by the force of the boy on the weight while he is holding it is:
\begin{choices}
\correctchoice 0
\choice 6.1J
\choice 40 J
\choice 60 J
\choice 90 J
\end{choices}

\question
A 2-kg object is moving at 3m/s. A 4-N force is applied in the direction of motion and then
removed after the object has traveled an additional 5m. The work done by this force is:
\begin{choices}
\choice 12 J
\choice 15 J
\choice18 J
\correctchoice 20 J
\choice 38 J
\end{choices}

%\includegraphics{raindrops.png}
%\question
%An open cart on a level surface is rolling without frictional loss through a vertical downpour of rain, as shown above. As the cart rolls, an appreciable amount of rainwater accumulates in the cart. The speed of the cart will
%\begin{choices}
%\choice increase because of conservation of momentum 
%\choice increase because of conservation of mechanical energy
%\correctchoice decrease because of conservation of momentum 
%\choice decrease because of conservation of mechanical energy
%\choice remain the same because the raindrops are falling perpendicular to the direction of the cart's motion
%\end{choices}

\question
A 1-kg block is lifted vertically 1m by a boy. The work done by the boy is about:
\begin{choices}
\choice 1 ft$\cdot$lb
\choice 1 J
\correctchoice 10 J
\choice 0.1J
\choice zero
\end{choices}

\question
The weight of an object on the moon is one-sixth of its weight on Earth. The ratio of the kinetic energy of a body on Earth moving with speed V to that of the same body moving with speed V on the moon is:
\begin{choices}
\choice 6:1
\choice 36:1
\correctchoice 1:1
\choice 1:6
\choice 1:36
\end{choices}

\newpage
\question
The amount of work required to stop a moving object is equal to:
\begin{choices}
\choice the velocity of the object
\correctchoice the kinetic energy of the object
\choice the mass of the object times its acceleration
\choice the mass of the object times its velocity
\choice the square of the velocity of the object
\end{choices}

\question
Which of the following bodies has the largest kinetic energy?
\begin{choices}
\choice Mass 3M and speed V
\choice Mass 3M and speed 2V
\correctchoice Mass 2M and speed 3V
\choice Mass M and speed 4V
\choice All four of the above have the same kinetic energy
\end{choices}

\question
The mechanical advantage of any machine is:
\begin{choices}
\choice the efficiency of the machine
\choice the work done by the machine
\choice the ratio of the work done by the machine to the work expended on it
\choice the ratio of the force exerted by the machine to the force applied to it
\choice the ratio of the force applied to the machine to the force exerted by it
\end{choices}

%efficiency question

\question
In raising an object to a given height by means of an inclined plane, as compared with raising the object vertically, there is a reduction in:
\begin{choices}
\choice work required
\choice distance pushed
\choice friction
\correctchoice force required
\choice value of the acceleration due to gravity
\end{choices}

%\textcolor{green}{
\question
A watt is:
\begin{choices}
\choice kg $\cdot$m/s$^3$
\choice kg$\cdot$m$^2$/s
\correctchoice kg$\cdot$m$^2$/s$^3$
\choice kg$\cdot$m/s
\choice kg$\cdot$m$^2$/s$^2$
\end{choices}

\question
A woman lifts a barbell 2.0m in 5.0 s. If she lifts it the same distance in 10 s, the work done by her is:
\begin{choices}
\choice four times as great
\choice two times as great
\correctchoice the same
\choice half as great
\choice one-fourth as great
\end{choices}

\newpage
\question
A person holds an 80-N weight 2m above the floor for 30 seconds. The power required to do this is:
\begin{choices}
\choice 80W
\choice40W
\choice 20W
\choice 10W
\correctchoice none of these
\end{choices}


\question
A golf ball is struck by a golf club and falls on a green three meters above the tee. The potential
energy of the ball is greatest:
\begin{choices}
\choice just before the ball is struck
\choice just after the ball is struck
\choice just after the ball lands on the green
\choice when the ball comes to rest on the green
\correctchoice when the ball reaches the highest point in its flight
\end{choices}

\question
For a block of mass m to slide without friction up the rise of height h shown, it must have a minimum initial kinetic energy of: \\
\includegraphics[scale = 0.6]{block.png}
\centering

\begin{choices}
\choice $gh$
\correctchoice $mgh$
\choice $gh/2$
\choice $mgh/2$
\choice $2mgh$
\end{choices}

\raggedright
%\textcolor{green}{
\question
A 0.50-kg block attached to an ideal spring with a spring constant of 80N/m oscillates on a horizontal frictionless surface. The total mechanical energy is 0.12 J. The greatest extension of the spring from its equilibrium length is:
\begin{choices}
\choice 1.5 x $10^{-3}$ m
\choice 3.0 x $10^{-3}$ m
\choice 0.039m
\correctchoice 0.054m
\choice 18m
\end{choices}

%\textcolor{green}{
\question
A 0.50-kg block attached to an ideal spring with a spring constant of 80N/m oscillates on a horizontal frictionless surface. The total mechanical energy is 0.12 J. The greatest speed of the block is:
\begin{choices}
\choice 0.15m/s
\choice 0.24m/s
\choice 0.49m/s
\correctchoice 0.69m/s
\choice1.46m/s
\end{choices}

\newpage
\question
For what contribution to physics was the 2013 Nobel Prize awarded?
\begin{choices}
\choice the discovery of accelerating expansion of the Universe 
\choice the development of multiscale models for complex chemical systems
\choice the discovery of photon
\correctchoice the discovery of the Higgs particle that contributes to our understanding of the origin of mass
\choice the discovery of a neutrino that travels faster than the speed of light \\[0.5in]
\end{choices}


\raggedright
\textbf{Part C:} Open Response\\
\textit{Answer both of the following open response questions. Partial credit will be given for your thought process, so be sure to answer all parts and show as much work as possible.}
\\[0.2in]

%\\setcounter{question}{0}



\question
The vertical position of an elevator as a function of time is shown.

\includegraphics{OR4.png}
\centering
\begin{subparts}
\subpart On the grid provided, graph the velocity of the elevator as a function of time.
\subpart Calculate the average acceleration for the time period t = 8 s to t = 10 s.
\subpart On the box provided that represents the elevator, draw a vector to represent the direction of this average acceleration.
\end{subparts}

\raggedright
\question 
A 2kg block initially hangs at rest at the end of two 1m strings of negligible mass. A 0.003kg bullet, moving horizontally with a speed of 1000m/s, strikes the block and becomes embedded in it. After the collision, the bullet/ block combination swings upward, but does not rotate. \\[0.3in]
\includegraphics{OR1.png}
\centering
\begin{subparts}
\subpart Calculate the kinetic energy of the bullet before the collision.
\subpart Calculate the speed $v$ of the bullet/ block combination just after the collision.
\subpart Assuming total mechanical energy is conserved, calculate the maximum vertical height above the initial rest position reached by the bullet/block combination. \\[1in]
\end{subparts}




\raggedright
\textbf{Part D:} Bonus\\
\textit{If you have enough time, consider the following bonus question:}
\\[0.2in]

\raggedright

A bullet of mass $m$ is moving horizontally with speed $v_o$ when it hits a block of mass 100m that is at rest on a horizontal frictionless table. The surface of the table is a height $h$ above the floor. After the impact, the bullet and the block slide off the table and hit the floor a distance $x$ from the edge of the table. \\
\includegraphics{OR2.png}
\centering

\raggedright
Derive an expression for the range achieved by the bullet-block system in terms of $m$, $h$, and $v_o$.

%\question
%
%
%\includegraphics[scale=1.2]{bonus.png}
%\centering


%\question
%A professor holds an eraser against a vertical chalkboard by pushing horizontally on it. He pushes with a force that is much greater than is required to hold the eraser. The force of friction exerted by the board on the eraser increases if he:
%\begin {choices}
%\choice pushes with slightly greater force
%\choice pushes with slightly less force
%\choice stops pushing
%\choice pushes so his force is slightly downward but has the same magnitude
%\choice pushes so his force is slightly upward but has the same magnitude
%\end{choices}


%\question
%An ideal spring, with a pointer attached to its end, hangs next to a scale. With a 100-N weight
%attached, the pointer indicates �40� on the scale as shown. Using a 200-N weight instead results
%in �60� on the scale. Using an unknown weight X instead results in �30� on the scale. The
%weight of X is:
%\begin {choices}
%\choice 10N
%\choice 20N
%\choice 30N
%\choice 40N
%\choice 50N
%\end{choices}

%\raggedright
%
%\question
%A Boston Red Sox baseball player catches a ball of mass $m$ that is moving toward him with
%speed $v$. While bringing the ball to rest, his hand moves back a distance $d$. Assuming constant
%deceleration, the horizontal force exerted on the ball by his hand is:
%\begin {choices}
%\choice $mv/d$
%\choice $mvd$
%\choice $mv^2/d$
%\choice $2mv/d$
%\choice $mv^2/(2d)$
%\end{choices}
%
%
%\question
%A block moves at 5.0m/s in the positive $x$ direction and hits an identical block, initially at
%rest. A small amount of gunpowder had been placed on one of the blocks. The explosion does
%not harm the blocks but it doubles their total kinetic energy. After the explosion the blocks
%move along the $x$ axis and the incident block has a speed of:
%\begin {choices}
%\choice 1.8m/s
%\choice 5.0m/s
%\choice 6.8m/s
%\choice 7.1m/s
%\choice 11.8m/s
%\end{choices}


\end{questions}






\end{document}